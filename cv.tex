\documentclass[11pt,a4paper]{moderncv}

\moderncvtheme[green]{classic}

\usepackage[utf8]{inputenc}

% personal data
\firstname{Enggar}
\familyname{Alfianto}
\title{Data Diri \newline \small{\texttt{\textbf{Homepage: {\color{web}\weblink{http://www.alfianto.com}}}}}}
%\address{Office 719 SEO\\ 851 S. Morgan St.}{Chicago, IL}
\extrainfo{Jojoran I Asri Blok B No 4 A\\ Mojo, Gubeng, Surabaya\\ Jawa Timur 60285}
\phone{+6283830244995}
\email{enggar@alfianto.com}
% \photo[64pt]{enggar}

\newcommand{\up}[1]{\ensuremath{^\textrm{\scriptsize#1}}}

% the ConTeXt symbol
\def\ConTeXt{%
  C%
  \kern-.0333emo%
  \kern-.0333emn%
  \kern-.0667em\TeX%
  \kern-.0333emt}
\definecolor{web}{rgb}{0.2,0.2,0.2}
%\definecolor{web}{rgb}{0.5,0.5,0.5}
%----------------------------------------------------------------------------------
%            content
%----------------------------------------------------------------------------------
\begin{document}
\maketitle

\section{\textbf{Data Diri}}
\cvitem{Nama Lengkap}{Enggar Alfianto}
\cvitem{Tempat}{Tuban}
\cvitem{Tanggal Lahir}{06 Juli 1987}
\cvitem{Agama}{Islam}
\cvitem{Alamat}{Jojoran 1 Asri Blok B no 4 A, Mojo, Gubeng, Surabaya, Jawa Timur}
\cvitem{Kode Pos}{60285}
% \cvitem{Alamat Surabaya}{Jl Jojoran 1 no. 65/E, Surabaya}
% \cvitem{Kode Pos}{60285}
\cvitem{Status Pernikahan}{Menikah}
\cvitem{No HP}{083830244995}
\cvitem{No HP}{085235019171}
\cvitem{email}{enggar.alfianto@gmail.com / enggar@alfianto.com}
\cvitem{Alamat Web}{http://www.alfianto.com}

\section{\textbf{Riwayat Pendidikan}}
\cventry{2012--2014}{\textbf{Pendidikan S2}}{Jurusan Sains Komputasi}{Institut Teknologi Bandung}{}{}
\cventry{2005--2011}{\textbf{Pendidikan S1}}{Jurusan Fisika Fakultas Sains dan Teknologi}{Universitas Airlangga}{Indonesia}{} %\newline{- Bidang Konsentrasi di bidang Fisika teoretik dan Koputasi,
% IPK total: 2.59/4.0.}  \newline
\cventry{2002--2005}{\textbf{Pendidikan Menengah Atas}}{}{SMAN 2}{Bojonegoro}{} 
\cventry{1999--2002}{\textbf{Pendidikan Menengah Pertama}}{}{SLTPN 1}{Rengel}{} 
\cventry{1993--1999}{\textbf{Pendidikan Dasar}}{}{SDN 2}{Prambonwetan, Rengel}{}{}


\section{\textbf{\textit{Background} Pendidikan}}
\cvitem{Modeling}{Sistem sains material, Komputasi dan Simulasi Sains Material, Sistem Dinamika Molekuler, Visualisasi dalam Sains}
\cvitem{Komputasi Parallel}{Pemrograman Parallel, Komputer Klaster/Komputer Performansi Tinggi}
\cvitem{Matematika}{Kalkulus, Fisika Matematik, ODE \& PDE, Fisika Statistik, Metode Numerik, ......}
\cvitem{Fisika Teoritik}{Fisika Dasar, Fisika Modern, Mekanika, Fisika Kuantum,
Fisika Zat Padat, Termodinamika, Fisika Inti, Fisika Optik, ......}
\cvitem{Komputasi}{Fisika Komputasi, Pemrograman Fisika, Komputer Cerdas Fisika, ......}
\cvitem{Elektronika \& Kontrol}{Elektronika Analog, Elektronika Digital, Mikroprosesor, Mikrokontroller ......}

\section{\textbf{Beasiswa}}
\cventry{2007}{\textbf{Beasiswa BBM}}{}{Universitas Airlangga}{Indonesia}{}


\section{\textbf{Publikasi}}
\cventry{2015}{\textbf{Implementasi Metode Density Functional Theory pada bahasa C untuk menghitung energi keadaan dasar berbagai atom}}{}{e-Jurnal Arus elektro Indonesia, Volume 1, No 3, 28 Desember 2015}{ISSN: 2443-2318}{halaman 1}
\cventry{2014}{\textbf{Acceleration of norm-conserving Pseudopotential Plane-Wave-Based DFT Calculation on GPU using CUDA}}{}{}{}{
 (\textit{September 2014, Volume 1, No.1 (Serial No.1)
Proceeding of Internattional Conference on Computation for Science and Technology})}
\cventry{2013}{\textbf{Pengaruh kadar gula dalam larutan terhadap daya serap Super Absorbent Polymer}}{Prosiding Seminar Nasional Fisika 2013}{ITB-Bandung}{ISBN:978-602-19655-5-9}{halaman 144}

\section{\textbf{Pengalaman}}
\subsection{\textbf{Dosen}}
\cventry{2015-2016}{\textbf{Dosen}} {Jurusan Sistem Komputer} {ITATS} {Indonesia}{}
\cventry{2015-2016}{\textbf{Dosen Luar Biasa}} {Jurusan Sistem Komputer} {Universitas Narotama} {Indonesia}{}
\subsection{\textbf{Asisten Dosen}}
\cventry{2013-2014}{\textbf{Assisten Laboratorium}} {Rekayasa Disain Material Komputasi} {ITB} {Indonesia}{}
\cventry{2007-2010}{\textbf{Asisten Dosen}}{Praktikum Fisika Dasar}{Universitas Airlangga} {Indonesia}{}
\cventry{2009-2010}{\textbf{Asisten Dosen}}{Praktikum Fisika Komputasi}{Universitas Airlangga} {Indonesia}{}
\cventry{2008-2009}{\textbf{Asisten Dosen}}{Praktikum Fisika Dasar}{STIKES RSI, Surabaya} {Indonesia}{} 
 
 
\subsection{\textbf{Penelitian}}
\cventry{2013-2014}{\textbf{Sains Komputasi}}{Fakultas MIPA}{ITB}{}{
 - Implementasi DFT Pada Pemrograman Berorientasi Objek untuk Menghitung Struktur Pita Energi dengan \textit{The Projector Augmented wave} (PAW) pada Karbon, Silikon dan Germanium (Tesis)
}
\cventry{2013-2014}{\textbf{Laboratorium Rekayasa Disain Material Komputasi}}{Teknik Fisika}{ITB}{}{
- Acceleration of norm-conserving Pseudopotential Plane-Wave-Based DFT Calculation on GPU using CUDA (\textit{September 2014, Volume 1, No.1 (Serial No.1)
Proceeding of Internattional Conference on Computation for Science and Technology})}

\cventry{2010-2011}{\textbf{Fisika Teori}}{Fakultas Sains dan Teknologi}{Airlangga University}{}{
- PERHITUNGAN NUMERIK ENERGI TOTAL KEADAAN DASAR UNTUK MOLEKUL SEDERHANA DENGAN DENSITY FUNCTIONAL THEORY, Penelitian ini digunakan untuk Skripsi Sebagai syarat kelulusan Sarjana.\newline
- Webpage: {\color{web} \weblink{http://www.unair.ac.id/}}} %\textasciitilde jyang06/statlab/StatLab.html}}}
\cventry{2009}{\textbf{ICTP- ITB Bandung}{}}{International Conference of Theoretical Physics}{OPI LIPI}{}{
- Perhitungan Energi Keadaan Magnetik Atom Golongan 3 Menggunakan Machikaneyama 2002, Sebagai paper yang dibawakan pada ICTP di ITB bandung}{}{}{}

\subsection{\textbf{Organisasi}}
\cventry{2003}{\textbf{Ketua Palang Merah Remaja}}{}{SMAN 2 Bojonegoro}{}{}{}{}
\cventry{2003}{\textbf{Ketua 1 MPK}}{}{SMAN 2 Bojonegoro}{}{}{}{}
\cventry{2005}{\textbf{Dirjen Optik}}{}{Himpunan Mahasiswa Fisika,}{Universitas Airlangga}{}{}
\cventry{2006}{\textbf{Ketua Mapanza Basic Learning}}{Pengkaderan Mahasiswa UKMAPANZA}{Universitas Airlangga}{}{}{}
\cventry{2007}{\textbf{Ketua Kuis Fisika Tingkat SMA se Jatim dan SMP se "Gerbangkertasusila"}}{Himpunan Mahasiswa Fisika}{Universitas Airlangga}{}{}
\cventry{2007}{\textbf{Sekretaris Umum}}{Unit Kegiatan Mahasiswa Peduli Penyalahgunaan NAPZA dan HIV/AIDS}{Universitas Airlangga}{}{}

%\section{\textbf{Completion Actuarial Exams}}
%\cventry{Nov, 2007}{\textbf{Exam P/1}}{Society of Actuaries}{}{}{}
%\cventry{May, 2008}{\textbf{Exam FM/1}}{Society of Actuaries}{}{}{}

\section{\textbf{Seminar dan Workshop}}
\cventry{Seminar}{3rd ICCST -\textbf{International Conference on Computation for Science and Technology}}{Bali, Indonesia}{}{}\\
\cventry{Workshop}{HPC -\textbf{Regional Workshop on Clustering for High Performance Computing}}{ICTP-LIPI-BATAN}{}{}\\
\cventry{Workshop}{Workshop on Proposal Development and Research Paper Authorship}{CRDF Global-HKI-Department of State USA}{}{}\\
\cventry{Seminar}{ICTP -\textbf{International Conference Of Theoretical Physics}}{ di ITB Bandung}{}{} \\
\cventry{Seminar}{ODSS -\textbf{One Day Scientific Seminar}}{ di Unair Surabaya}{}{} \\
\cventry{Workshop}{CMD -\textbf{Computational Material Design}}{ di ITB Bandung dengan instruktur dari Osaka University, Jepang}{}{} \\
\cventry{Workshop}{TOT -\textbf{Training Of Trainer}}{ di Unair, Surabaya}{}{} 


\section{\textbf{Bahasa}}
\cvlanguage{Indonesia}{Bahasa Induk}{}
\cvlanguage{Inggris}{Bahasa Asing}{
TOEFL(PBT-ITB) SKOR: 100/160,
}

\section{\textbf{Keahlian Komputer}}
% \cventry{OS}{TOT -\textbf{Training Of Trainer}}{ di Unair, Surabaya}{}{}
\cvcomputer{\tetxbf{OS}}{Linux/Unix, Windows,}{\textbf{Simulasi}}{Gromac, Quantum Espresso, HilapW, CPA-2012}
\cvcomputer{\textbf{Pengolah Data}}{SPSS, MS excell,,Matlab, Maple, Matematica}{\textbf{Pengetikan}}{\LaTeX, Microsoft Office}
\cvcomputer{\textbf{Pemrogram-an}}{Pascal, Delphi, C, Java, HTML, SQL}{}{}


% \photo[100pt]{a.jpeg}
%\begin{figure}[hbp]
%	\centering
%		\includegraphics[width=0.50\textwidth]{E:/enggar/CV/keren/enggar/a.jpg}
%	\caption{Foto Terbaru}
%	\label{fig:a}
%\end{figure}


\section{\textbf{Bidang yang diminati}}
\cvlistitem{Pemrograman, Modeling dan Simulasi, FIsika Plasma, Plasma modelling }
%\cvlistitem{Time Series Analysis}
%\cvlistitem{Spatial Statistics}
%\cvlistitem{Stochastic Process}
%\cvlistitem{Data/Text Mining}
%\cvlistitem{Boundary Value Problems}
%\cvlistitem{Numerical Analysis}
%\cvlistitem{Parallel Computing}
\end{document}
